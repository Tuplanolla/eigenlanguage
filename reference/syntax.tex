\documentclass{article}
\pagestyle{empty}
\usepackage[
margin=0.5in,
a4paper
]{geometry}
\usepackage{listings, multicol}
\lstset{
aboveskip=0pt,
literate=
{=}{$=$} 1
{->}{$\rightarrow$} 2
{<-}{$\leftarrow$} 2
{<->}{$\longleftrightarrow$} 3
{...}{$\cdots$} 2
{`}{${}^{\scriptscriptstyle\backslash}$} 1
{,}{${}_{\scriptscriptstyle/}$} 1
{\#'}{`} 1
{'\#}{'} 1
{\#"}{``} 1
{"\#}{''} 1,
mathescape
}

\begin{document}
\begin{multicols} 2
[\section*{\huge Syntax of Eigenlanguage}\hrule]

\bigskip\subsection*{Application}
% f a
\begin{lstlisting}
$f$ $a$
\end{lstlisting}
Applies the function \lstinline{$f$} to its argument \lstinline{$a$}.

\bigskip\subsection*{Singleton}
\begin{lstlisting}
()
\end{lstlisting}
Represents the only value of a unit type.

\bigskip\subsection*{Left-Recursive Group}
% (x_1 ...)
\begin{lstlisting}
($x_1$ ...)
\end{lstlisting}
Builds pairs of expressions \lstinline{$x$} from right to left.
For example \lstinline{($x_1$ $x_2$ $x_3$)}
yields the code \lstinline{((() $x_1$) $x_2$) $x_3$}.

\bigskip\subsection*{Right-Recursive Group}
% [x_1 ...]
\begin{lstlisting}
[$x_1$ ...]
\end{lstlisting}
Builds pairs of expressions \lstinline{$x$} from left to right.
For example \lstinline{[$x_1$ $x_2$ $x_3$]}
yields the code \lstinline{$x_1$ ($x_2$ ($x_3$ ()))}.

\bigskip\subsection*{Code as Data}
% `x
\begin{lstlisting}
`$x$
\end{lstlisting}
Treats the expression \lstinline{$x$} as a value.
Works recursively if repeated.
For example \lstinline{`($x_1$ ($f$ $a_1$ $a_2$) $x_3$)}
produces the data \lstinline{($x_1$ $y$ $x_3$)}.

\bigskip\subsection*{Data as Code}
% ,x
\begin{lstlisting}
,$x$
\end{lstlisting}
Treats the value \lstinline{$x$} as an expression.
For example \lstinline{``($x_1$ ,($f$ $a_1$ $a_2$) $x_3$)}
produces the data \lstinline{($x_1$ $y$ $x_3$)}.

\bigskip\subsection*{Binding}
% = (y_1 x_1
%    ... ...)
%   z
\begin{lstlisting}
= ($y_1$ $x_1$
   ... ...)
  $z$
\end{lstlisting}
Binds symbols \lstinline{$y$} to expressions \lstinline{$x$}
inside expression \lstinline{$z$} and every \lstinline{$x$}.

\bigskip\subsection*{Function}
% -> p z
\begin{lstlisting}
-> $p$ $z$
\end{lstlisting}
Defines an anonymous function
and binds its parameter \lstinline{$p$} to its argument
inside \lstinline{$z$}.

\bigskip\subsection*{Nested Functions}
% -> (p_1 ...) z
\begin{lstlisting}
-> ($p_1$ ...) $z$
\end{lstlisting}
Defines an anonymous function and
binds its parameters \lstinline{$p$} to its arguments inside \lstinline{$z$}.
For example \lstinline{-> ($p_1$ $p_2$ $p_3$) $z$}
is equivalent to \lstinline{-> $p_1$ (-> $p_2$ (-> $p_3$ $z$))}.

\bigskip\subsection*{Name Qualification}
% m/e
\begin{lstlisting}
$m$/$e$
\end{lstlisting}
Resolves to the exported symbol \lstinline{$e$} from module \lstinline{$m$}.

\columnbreak

\bigskip\subsection*{Module}
% <-> (m p_1 ...) (
% -> (e_1
%     (= (b_2 e_2
%         b_3 e_3
%         ... ...)
%      ... ...)
%     ...)
% <- (i_1
%     (i_2 a_2_1 ...)
%     (= (c_3 i_3
%         c_4 (i_4 a_4_1 ...)
%         ... ...)
%      ... ...)
%     ...)
% y_1 x_1
% ... ...
% )
\begin{lstlisting}
<-> ($m$ $p_1$ ...) (
-> ($e_1$
    (= ($b_2$ $e_2$
        $b_3$ $e_3$
        ... ...)
     ... ...)
    ...)
<- ($i_1$
    ($i_2$ $a_{2, 1}$ ...)
    (= ($c_3$ $i_3$
        $c_4$ ($i_4$ $a_{4, 1}$ ...)
        ... ...)
     ... ...)
    ...)
$y_1$ $x_1$
... ...
)
\end{lstlisting}
Declares the module \lstinline{$m$} with parameters \lstinline{$p$}.
Exports symbols \lstinline{$e$},
imports modules \lstinline{$i$} and
binds symbols \lstinline{$y$} to expressions \lstinline{$x$}
inside every \lstinline{$e$}, every \lstinline{$a$} and every \lstinline{$x$}.
Gives some exports \lstinline{$e$} aliases \lstinline{$b$},
some imports \lstinline{$i$} aliases \lstinline{$c$} and
some imports \lstinline{$i$} arguments \lstinline{$a$}.

\bigskip\subsection*{Number}
\begin{lstlisting}
+18_12
\end{lstlisting}
Represents the number $20$, which
is $18$ in base $12$.

\bigskip\subsection*{Character}
% 'T'
\begin{lstlisting}
#'T'#
\end{lstlisting}
Is the $20$th character of the alphabet.

\bigskip\subsection*{String}
% "This text
% is arbitrary."
\begin{lstlisting}
#"This text
is arbitrary."#
\end{lstlisting}
Contains text with escape sequences.

\bigskip\subsection*{External String}
% <- "arbitrary_file.text"
\begin{lstlisting}
<- #"arbitrary_file.text"#
\end{lstlisting}
Reads a file that contains text without escape sequences.

\bigskip\subsection*{Syntactic Comment}
% %arbitrary-expression
\begin{lstlisting}
%arbitrary-expression
\end{lstlisting}

\bigskip\subsection*{Line Comment}
\begin{lstlisting}
% This text is arbitrary.
\end{lstlisting}

\bigskip\subsection*{Block Comment}
\begin{lstlisting}
%%
This text
is arbitrary.
%%
\end{lstlisting}

\end{multicols}
\end{document}
